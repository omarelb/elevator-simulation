\section{Model of the System}

This section introduces the model of the elevator system studied here. The modelling approach is similar to the one used by \cite{yuan_2008}. We use this model in all further analysis. The system dynamics are parameterized as follows:

\begin{itemize}
    \item Number of floors: 5.
    \item Number of elevator cars: 1. We consider only a single elevator to simplify the problem.
    \item Elevator floor time (time it takes for an elevator to pass a floor at full speed): ??.
    \item Elevator stop time (time it takes for the elevator to decelarate, open and close the doors, and accelerate again): ??.
    \item Elevator load time (the time it takes for a passenger to enter or exit the elevator): 1 second.
    \item Floor height. Set to ?? here.
    \item Capacity of the cars: 8 passengers.
    \item Maximum number of passengers waiting for an elevator: 2 passengers.
\end{itemize}

We model the elevator system as a discrete-time system. In this way, we can model the environment as a Markov Decision Process with a finite number of states and actions. Using this approach, the states and actions can be fully specified.

The size of the state space can be changed by varying the number of floors, number of elevator cars, capacity of the cars and the maximum number of passengers. The parameters have been chosen such that the state space is finite and reasonably sized. This is of great importance when using algorithms in which it is necessary to store values for each state. If the state space is too large, running time and memory capacity can become intractable. We would need to approximate the value functions. 

We use several assumptions to simplify the model further. We assume that the maximum number of passengers waiting for an elevator is 2. If another passenger arrives at a floor where 2 passengers are already waiting, we treat it as if there are still only 2 passengers waiting. We do this to limit the size of the state space.

It is not possible to observe the full state of the system. After a button is pressed, the elevator does not know if another passenger arrived after the first one. One way to deal with this is assuming \textit{omniscience}. We assume that in every state, the elevator controller knows how many passengers are waiting at a floor, when they arrived and where they are going. Although this is not a particularly realistic assumption, it simplifies analysis of the system.

\subsection{Traffic and Passenger Arrival}

It is important to take into account the traffic profile at a time. General building traffic profiles have been identified \cite{elevator_dynamics}. Four important profiles are up-peak, down-peak, inter-floor, and lunchtime. We will concern ourselves only with the down-peak traffic profile. Down-peak is a traffic pattern in which passengers are primarily moving down to the ground floor. An example is people going home at the end of a business day in an office building. We will assume every arriving passenger wants to go to the ground floor.

We model the arrival of passengers as a Poisson process with rate parameter $\lambda$ being the expected number of people arriving at a floor each minute. The rate can vary across floors. We set it to ??. For every floor, once a passenger arrives, the next passenger will arrive after some amount of time. The amount of time is drawn from an exponential distribution with rate $\lambda$.

\subsection{State and Action Space}

Let $N$ be the number of floors in the building, including the ground floor.

A state $s$ is defined by several variables:

\[
% s = \begin{bmatrix} c, & w, & p, & o, & d \end{bmatrix}^\top
s = \begin{bmatrix} c, & p, & o, & d \end{bmatrix}^\top
    % s = [c,\ p,\ o,\ d]^\top    
\]

where

\begin{itemize}
    \item $c$ is an $N-1$ length vector indicating the number of people waiting at each floor excluding the ground floor. The entries of $c$ take values in \{0, 1, 2\}.
    % \item $w$ is an $N - 1$ length vector indicating the mean waiting time of people on each floor. Since this is a continuous quantity, we discretize it into bins indicating short, medium and long waiting times. [0, ??) = low, [??, ??) = medium, [??, $\infty$) = high
    \item $p \in \{0, 1, \dots, N-1\}$ indicates at which floor the elevator is locatesd.
    \item $o \in \{0, 1, \dots, \text{elevator\_capacity}\}$ indicates how many passengers are occupying the elevator.
    \item $d \in \{-1, 0, 1\}$ indicates the direction of the elevator. In ascending order, the values indicate a downward movement, no movement, and an upward movement respectively.
\end{itemize}

The cardinality of the state space is 
\begin{align*}
    &(\text{floor\_capacity} + 1)^{N - 1} \cdot N \cdot (\text{elevator\_capacity + 1}) \cdot \text{num\_directions}\\
    & = 3 ^ {4} \cdot 5 \cdot 9 \cdot 3 = 10935
\end{align*}

which is a reasonably small size.

Now that we have the state space defined, we can move on to defining the actions we can take. What action the elevator is able to take will depend on the state of the system.

We define the actions as follows:

\begin{itemize}
    \item If the elevator is moving $(d \neq 0)$:
        \begin{itemize}
            \item Stop at next floor.
            \item Continue past next floor.
        \end{itemize}

    \item If the elevator is not moving $(d = 0)$:
        \begin{itemize}
            \item Go up
            \item Go down
        \end{itemize}
\end{itemize}

There are additional restrictions on what actions can be taken. When at the bottom and top floors, we can not choose the action go down and go up respectively. We cannot continue past the bottom and top floors. We can not turn in a single action. If the current direction is $1$, for example, we have to first stop the elevator before we can set it to $-1$. Taking into account passenger expectations, a car cannot turn until it has served all the calls in its present direction.

\subsection{Performance Measures and Reward}

We need a way to measure the performance of the method we are applying. The goal is to minimize some function of passengers' waiting time. We consider the average passenger waiting time, which is generally considered a primary objective \cite{elevator_dynamics}. The waiting time of a passenger is defined as the time between the passenger's arrival at the floor and the passenger's entry into a car. Other possible performance measures are system time and the fraction of passengers waiting more than $T$ seconds, where $T$ is typically 60. System time is defined as the waiting time combined with the passenger's travel time.

We want to define a reward such that maximizing this reward will lead to a lower average waiting time. We do this by defining the reward function

\[
    r(s, a, s') = - \sum_{i=0}^{N-1} c_i    
\]

where $c_i$ is the $i$th element of $c$. $s'$ is the next state observed after taking action $a$ in state $s$. {\color{red} (MAKE NOTATION SECTION?)}. In other words,
the reward is higher with less people in the system. {\color{red} (ADJUST REWARD TO TAKE INTO ACCOUNT WAITING TIME?)}. A reward $r_t$ at timestep $t$ is observed after taking an action in timestep $t - 1$. 

% DISCRETIZATION OF WAITING TIME